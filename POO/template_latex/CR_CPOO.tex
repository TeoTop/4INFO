
\documentclass[a4paper,12pt]{article}

\usepackage{ucs}
\usepackage[utf8x]{inputenc}
%\usepackage[latin1]{inputenc}
\usepackage[T1]{fontenc}

\usepackage[french]{babel}

\pagestyle{plain}

\usepackage{graphicx}
\usepackage{subfigure}
\DeclareGraphicsExtensions{.pdf,.eps,.jpg,.png,.gif}

\usepackage{color}
\definecolor{grey}{rgb}{0.9,0.9,0.9}
\definecolor{teal}{rgb}{0.0,0.5,0.5}
\definecolor{violet}{rgb}{0.5,0,0.5}

\usepackage{listings}
\usepackage{listingsutf8}
\lstloadlanguages{[Visual]C++}
\lstdefinestyle{listing}{
  language=Java,
  captionpos=t,
  inputencoding=utf8/latin1,
  extendedchars=true,
  resetmargins=true,
  xleftmargin=-60pt,
  xrightmargin=-70pt,
%  frame=single,
  numbers=left,
  numberstyle=\tiny,
  numbersep=5pt,
  breaklines=true,
  breakatwhitespace=true,
  showspaces=false,
  showstringspaces=false,
  showtabs=false,
  tabsize=2,
  basicstyle=\footnotesize\ttfamily,
  backgroundcolor=\color{grey},
  keywordstyle=\color{blue}\bfseries,
  commentstyle=\color{teal},
  identifierstyle=\color{black},
  stringstyle=\color{red},
  numberstyle=\color{violet},
}
\lstset{style=listing}

%%%%%%%%%%%%%%%%%%%%%%%%%%%%%%%%%%%%%%%%%%%%%%%%%%%%%%%%%%%%

\author{
  Théo \textsc{CHAPON}, Hassan \textsc{EL OMARI ALAOUI} \\ \\
  INSA de Rennes \\
  4INFO, groupe 1.2
}

\title{Compte Rendu TP1 CPOO}

\begin{document}
\maketitle{TP1 : Conception de la classe carte}

section{Etat du TP}

Le projet est terminé, fonctionel et testé. Aucuns problèmes et aucunes difficultés n'est à déplorer.


section{Objectif}

Ce TP avait pour but d'avoir une première approche avec Microsoft Visual 2013 et le langage C++. Il nous a été demandé de 
compléter un programme permettant de simuler un jeu de bataille. La classe principale nous a été donnée. Nous devions donc 
créer la classe carte.


section{Réalisation}

La classe carte se compose donc de deux parties : la première est le header, classe.h, qui permet de définir les composantes
et fonctions de la classe. La seconde, classe.cpp, permet de coder les fonctions de la classe carte.

Pour réaliser le jeu de carte, nous avons utilisé le principe d'une liste simplement chainée. Les cartes se composent donc d'un
propriétaire d'une couleur et d'une valeur. De plus, pour construire la liste chainée, nous avons créé un atribut 'succ' qui
est un pointeur sur carte permettant de faire référence à la carte suivante dans le paquet. Nous avons aussi mis en place 4
attributs statics qui permettent de faire référence à la première carte et la dernière de chaque paquet (2 pour une bataille).
La carte de début et de fin d'un paquet permettent de définir les limites d'une liste.

Nous avons aussi créé et complété les fonctions présentent dans le déroulement du code principale. A cela, nous avons rajouté
des fonctions supplémentaires comme 'void _change(bool = true)', 'bool _equal(Carte)' et 'Carte* suc()'. Elles permettent
respectivment de changer le propriétaire de la carte, de savoir si la carte est la même que celle passée en paramètre et
de rendre la carte qui suit celle sélectionnée dans le paquet.


\lstinputlisting[caption=Définition de la class carte en C++]{carte.h}
\lstinputlisting[caption= Class carte en C++]{carte.cpp}

\end{document}
