\documentclass[a4paper,12pt]{article}

\usepackage{ucs}
\usepackage[utf8x]{inputenc}
%\usepackage[latin1]{inputenc}
\usepackage[T1]{fontenc}

\usepackage[french]{babel}

\pagestyle{plain}

\usepackage{graphicx}
\usepackage{subfigure}
\DeclareGraphicsExtensions{.pdf,.eps,.jpg,.png,.gif}

\usepackage{color}
\definecolor{grey}{rgb}{0.9,0.9,0.9}
\definecolor{teal}{rgb}{0.0,0.5,0.5}
\definecolor{violet}{rgb}{0.5,0,0.5}

\usepackage{listings}
\usepackage{listingsutf8}
\lstloadlanguages{[Visual]C++}
\lstdefinestyle{listing}{
  language=Java,
  captionpos=t,
  inputencoding=utf8/latin1,
  extendedchars=true,
  resetmargins=true,
  xleftmargin=-60pt,
  xrightmargin=-70pt,
%  frame=single,
  numbers=left,
  numberstyle=\tiny,
  numbersep=5pt,
  breaklines=true,
  breakatwhitespace=true,
  showspaces=false,
  showstringspaces=false,
  showtabs=false,
  tabsize=2,
  basicstyle=\footnotesize\ttfamily,
  backgroundcolor=\color{grey},
  keywordstyle=\color{blue}\bfseries,
  commentstyle=\color{teal},
  identifierstyle=\color{black},
  stringstyle=\color{red},
  numberstyle=\color{violet},
}
\lstset{style=listing}

\author{
  Théo \textsc{CHAPON}, Hassan \textsc{EL OMARI ALAOUI} \\ \\
  INSA de Rennes \\
  4INFO, groupe 1.1
}

\title{Compte Rendu TP2 CPOO}

\begin{document}
\maketitle{TP1 : Conception de la classe chaine et surcharge}

section{Etat du TP}

Le projet est terminé, fonctionel et testé. Une incompréhension a été rencontrée lors de l'utilisation de l'attribut 'friend'.
Mais celle-ci a été résolue après discution.


section{Objectif}

Ce TP avait pour but d'effectuer de la surcharge d'opérateur. En plus de cela, nous avons pu travailler les pointeurs et la
référence sur objet. Nous avons aussi pu tester les attributs 'friend' et 'inline'. Pour ce faire nous devions créer une 
classe chaine permettant de gérer une chaine de caractères ainsi que certaines opérations comme la concaténation.


section{Réalisation}

Nous devions réaliser une classe permettant de gérer les chaines de caratères. Pour ce faire nous avons décidé d'utiliser 
un tableau de caratères comme attribut. Nous avons géré la longueur des chaines grâce au caractère de fin de chaine '\0'.
Pour faciliter la gestion de la longueur, nous avons créé une méthode permettant de retourner la longueur d'une chaine.

Pour commencer, nous avons créé trois constructeurs permettant de créer une chaine par défault (contructeur vide), de
créer une chaine de caractères à partir d'un tableau de caractères (pointeur sur char), de créer une chaine à partir d'une 
autre chaine (argument de type chaine passé par référence) et enfin un destructeur. 

De plus, nous devions produire des méthodes permettant de gérer les chaines de caractères. Nous avions deux types de méthodes
à traiter; les méthodes de surcharge pour utiliser les opérateurs pour effectuer des comparaisons (<, >, == etc.),
pour faire une concaténation (+, +=) ou encore sélectionner un caractère indéxé de la chaine ([]). Nous avions aussi les 
méthodes normales comme les getters et les setters, les deux méthodes 'sous-chaine' dont la fonction était de récupérer une
partie de la chaine à partir de deux indexes ou de deux caractères.

A cela nous avons aussi ajouté la surcharge au niveau de l'affichage d'une chaine (outstream : surcharge de l'opérateur <<).


\lstinputlisting[caption=Définition de la class carte en C++]{chaine.h}
\lstinputlisting[caption= Class carte en C++]{chaine.cpp}

\end{document}
