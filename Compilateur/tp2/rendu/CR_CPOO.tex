\documentclass[a4paper,12pt]{article}

\usepackage{ucs}
\usepackage[utf8x]{inputenc}
%\usepackage[latin1]{inputenc}
\usepackage[T1]{fontenc}


\pagestyle{plain}

\usepackage{graphicx}
\usepackage{subfigure}
\DeclareGraphicsExtensions{.pdf,.eps,.jpg,.png,.gif}

\usepackage{color}
\definecolor{grey}{rgb}{0.9,0.9,0.9}
\definecolor{teal}{rgb}{0.0,0.5,0.5}
\definecolor{violet}{rgb}{0.5,0,0.5}

\usepackage{listings}
\usepackage{listingsutf8}
\usepackage{amsmath}
\lstloadlanguages{[Visual]C++}
\lstdefinestyle{listing}{
  language=Java,
  captionpos=t,
  inputencoding=utf8/latin1,
  extendedchars=true,
  resetmargins=true,
  xleftmargin=-60pt,
  xrightmargin=-70pt,
%  frame=single,
  numbers=left,
  numberstyle=\tiny,
  numbersep=5pt,
  breaklines=true,
  breakatwhitespace=true,
  showspaces=false,
  showstringspaces=false,
  showtabs=false,
  tabsize=2,
  basicstyle=\footnotesize\ttfamily,
  backgroundcolor=\color{grey},
  keywordstyle=\color{blue}\bfseries,
  commentstyle=\color{teal},
  identifierstyle=\color{black},
  stringstyle=\color{red},
  numberstyle=\color{violet},
}
\lstset{style=listing}

\author{
  Théo \textsc{CHAPON}, Alexandre \textsc{Audinot} \\ \\
  INSA de Rennes \\
  4INFO, groupe A
}

\title{Compte Rendu de compilation TP2}

\begin{document}
\maketitle{TP2 : Conception de la classe chaine et surcharge}


\section{Nouvelle grammaire}

<Super>    --> <Expr> eof \\
<Expr>     --> <Termb> <Expr2> \\
<Expr2>    --> "ou" <Termb> <Expr2> | $\epsilon$ \\
<Termb>    --> <Facteurb> <Termb2> \\
<Termb2>   --> "et" <Facteurb> <Termb2> | $\epsilon$ \\
<Facteurb> --> <Relation> | parouv <Expr> parferm | si <Expr> alors <Expr> sinon <Expr> fsi \\
<Relation> --> ident <Op> ident, \\
<Op>       --> "=" | "<>" | "<" | ">" | ">=" | "<=" \\


\section{Equation sur les règles}
\begin{align*}
premier(Super)&= premier(\text{Expr eof}) \\
        &= premier(\text{Expr}) \\
        &= \{\text{"(", "si", "ident"}\}
\end{align*}

\begin{align*}
premier(Expr)   &= premier(\text{Termb Expr2}) \\
        &= premier(\text{Termb}) \\
        &= \{\text{"(", "si", "ident"}\}  
\end{align*}

\begin{align*}
premier(Expr2)  &= premier(\text{"ou" Termb Expr2}) \cup \{\epsilon\} \\
        &= \{"ou"\}
\end{align*}

\begin{align*}
premier(Termb)  &= premier(\text{Facteurb Termb2}) \\
        &= premier(Facteurb) \\
        &= \{\text{"(", "si", "ident"}\}
\end{align*}

\begin{align*}
premier(Termb2) &= premier(\text{"et" Facteurb Termb2}) \cup \{\epsilon\} \\
        &= \{"et"\}
\end{align*}

\begin{align*}
premier(Facteurb) &= premier(Relation) \cup premier(\text{"parouv" Expr "parferm"}) \\
   & \cup premier(\text{"si" Expr "alors" Expr "sinon" Expr "fsi"}) \\
          &= premier(Relation) \cup \{"parouv"\} \cup \{"si"\} \\
          &= \{\text{"(", "si", "ident"}\}
\end{align*}

\begin{align*}
premier(Relation) &= premier(\text{"ident" Op "ident"}) \\
          &= \{"ident"\}
\end{align*}

\begin{align*}
premier(Op)   &= premier("=") \cup premier("<>") \cup premier("<") \cup premier(">") \\
    & \cup premier(">=") \cup premier("<=") \\
        &= \{\text{"=", "<>", "<", ">", ">=", "<="}\}
\end{align*}

\begin{align*}
suivant(super) &= \emptyset
\end{align*}

\begin{align*}
suivant(Expr) &= \{"eof"\} \cup \{"parferm"\} \cup \{"alors"\} \cup \{"sinon"\} \cup \{"fsi"\} \\
        &= \{\text{eof, ")", "alors", "sinon", "fsi"}\}
\end{align*}

\begin{align*}
suivant(Expr2)  &= suivant(Expr) \cup suivant(Expr2) \cup \{\epsilon\} \\
        &= suivant(Expr) \\
        &= \{\text{eof, ")", "alors", "sinon", "fsi"}\}
\end{align*}

\begin{align*}
suivant(Termb)  &= premier(Expr2) \cup suivant(Expr2) \cup \{\epsilon\} \\
        &= premier(Expr2) \cup suivant(Expr2) \\
        &= \{\text{"ou", eof, ")", "alors", "sinon", "fsi"}\}
\end{align*}

\begin{align*}
suivant(Termb2) &= premier(Termb) \cup suivant(Termb2) \cup \{\epsilon\} \\
        &= premier(Termb) \\
        &= \{\text{"ou", eof, ")", "alors", "sinon", "fsi"}\}
\end{align*}

\begin{align*}
suivant(Facteurb)   &= premier(Termb2) \cup suivant(Termb2) \cup \{\epsilon\} \\
          &= premier(Expr2) \cup suivant(Expr2) \\
          &= \{\text{"et", "ou", eof, ")", "alors", "sinon", "fsi"}\}
\end{align*}

\begin{align*}
suivant(Relation) &= suivant(Facteurb) \\
          &= \{\text{"et", "ou", eof, ")", "alors", "sinon", "fsi"}\}
\end{align*}

\begin{align*}
suivant(Op) &= \{"ident"\}
\end{align*}

\begin{align*}
null(Expr)       &= false  \\
null(Expr)        &= false  \\
null(Expr2)       &= true \\
null(Termb)       &= false \\
null(Termb2)      &= true \\
null(Facteurb)    &= false \\
null(Relation)    &= false \\
null(Op)          &= false
\end{align*}


\section{Vérification grammaire LL1}


\noindent Super:

  Pas d'intersection

  null(Super) = false

  \medskip

\noindent Expr:

  Pas d'intersection

  null(Expr) = false

  \medskip

\noindent Expr2:

   premier("ou") $\cap$ premier($\emptyset$) = $\emptyset$

   null(Expr2) = true

   premier(Expr2) $\cap$ suivant(Expr2) = $\emptyset$

\medskip

\noindent Termb:

   Pas d'intersection

   null(Super) = false

\medskip

\noindent Termb2:

   premier("et") $\cap$ premier($\emptyset$) = $\emptyset$

   null(Termb2) = true

   premier(Termb2) $\cap$ suivant(Termb2) = $\emptyset$

\medskip

\noindent Facteurb:

   premier(Relation) $\cap$ premier("parouv") $\cap$ premier("si") = $\emptyset$

   null(Facteurb) = false

\medskip

\noindent Relation:

   Pas d'intersection

   null(Relation) = false

\medskip

\noindent Op:

   premier("=") $\cap$ premier("<>") $\cap$ premier("<") $\cap$ premier(">") $\cap$ \\ premier(">=") $\cap$ premier("<=") = $\emptyset$

   null(Op) = false

\medskip

\noindent Toutes les règles respectent bien les vérifications. C'est une grammaire LL1.

\section{Arbre concret}

\lstinputlisting[caption=Arbre concret sur l'exemple "t > y et x = y"]{arbre_concret.txt}
\medskip


\section{Jeux de tests}

\subsection{Non-terminaux inaccessibles}

\noindent Test (sans Relation -> ident Op ident) : non-terminaux inaccessibles.

\noindent Dans ce cas, nous avons mis la règle "Relation" à vide pour empécher d'atteindre le terminaux de la règle "Op". 

\lstinputlisting[caption= Non-terminaux inaccessibles]{test1.txt}

\subsection{Non-terminaux non productifs}

\noindent Test (sans Expr2 ->) : non-terminaux qui ne sont pas productifs.

\noindent Dans ce cas, nous avons retirer la partie vide de la règle "Expr2" afin de boucler sur la clause "ou" de la grammaire et ainsi obliger celle-ci à être non-terminal.

\lstinputlisting[caption= Non-terminaux non productifs]{test2.txt}


\subsection{Grammaire non LL1}

\noindent Test (Expr ancienne grammaire) : grammaire non LL1.

\noindent Dans ce cas, nous avons réutiliser la première grammaire sur la règle "Expr" afin de placer la dérivation à gauche et non à droite comme convenue pour qu'une grammaire soit LL1.

\lstinputlisting[caption= Grammaire non LL1]{test3.txt}


\subsection{Grammaire LL1}

\noindent Cas final du TP.

\lstinputlisting[caption= Grammaire LL1]{test4.txt}


\subsection{Test de mots reconnus}



\lstinputlisting[caption= {$a=b$}]{exemple1.txt}


\lstinputlisting[caption= {$a=b\text{ et }a<>c$}]{exemple3.txt}



\lstinputlisting[caption= {$\text{si }a=b\text{ alors }a=b\text{  sinon }a=c\text{  fsi}$}]{exemple2.txt}

\medskip
\subsection{Test de mots non reconnus}

\noindent a et b

\noindent Failure("No derivation")

\medskip

\noindent a(b) = c

\noindent Failure("No derivation")


\end{document}
